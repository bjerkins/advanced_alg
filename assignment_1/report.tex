\documentclass[12pt]{article}
\usepackage[T1]{fontenc}
\usepackage[utf8]{inputenc}
\usepackage{url}
\usepackage{enumerate}
\usepackage[top=3cm, bottom=3cm]{geometry}
\usepackage{graphicx}
% \usepackage{enumitem}
\usepackage{natbib}
\usepackage{listings}
\usepackage{float}
\usepackage{amsmath}
\usepackage{color}
\bibpunct{[}{]}{,}{a}{}{;}
\setcitestyle{super}

% Variables
\newcommand{\assignmentname}{Assignment 1}
\newcommand{\coursename}{Advanced Algorithms and Data Structures}
\newcommand{\studentname}{Bjarki Madsen (lch929) - Michael Bang (tqg432)}
\newcommand{\department}{Department of Computer Science}
\newcommand{\institution}{Copenhagen University}
\newcommand{\location}{Copenhagen, Denmark}

\begin{document}

\renewcommand\refname{References}

\title{\assignmentname \\ {\Large {\textsc \coursename}}}
\author{
        \studentname \\
        \department \\
        \institution \\
        \location
}
\date{\today}

\maketitle
\thispagestyle{empty}

\pagebreak

\section*{Exercise 1}

  Figure \ref{fig:e1_a_solution} shows a b-flow with maximum flow value of 9 for the first proposed graph from the assignment. The second proposed graph has no b-flow. The reason is that $v_1$ wants to have at least $3 + c(v_2, v_1)$ after giving away $f(v_1, v_3)$. In order for that to happen the $f(v_2, v_1)$ has to equal to $c(v_2, v_1)$. After the $f(v_2, v_1)$, $v_2$ wants to end up with 1 in demand, meaning it needs a flow value of 4 to itself but that is impossible because the maximum flow to $v_2$ is equal to $c(v_3, v_2)$, or 2 and therefore is unable to fulfill $f(v_2, v_1) = c(v_2, v_1)$.

  \begin{figure}[h]
    \centering
      \includegraphics[width=0.6\textwidth]{figures/e1_a_solution}
    \caption{Graph showing b-flow with flow value 9}
    \label{fig:e1_a_solution}
  \end{figure}

\section*{Exercise 2}
\subsection*{Exercise 2.1}

  Tables \ref{table:values_xvf} and \ref{table:values_zfg} show the values for $x_{vf}$ and $z_{fg}$, respectively. The number of breakpoints are 13.

  \begin{table}[h]
    \centering
    \begin{tabular}{llllllll}
                                    & \textbf{v1} & \textbf{v2} & \textbf{v3} & \textbf{v4} & \textbf{v5} & \textbf{v6} & \textbf{v7} \\ \cline{2-8}
    \multicolumn{1}{c|}{\textbf{a}} & 0           & 0           & 1           & 0           & 1           & 1           & 0           \\
    \multicolumn{1}{l|}{\textbf{b}} & 1           & 0           & 0           & 0           & 0           & 1           & 0           \\
    \multicolumn{1}{l|}{\textbf{c}} & 1           & 1           & 1           & 0           & 0           & 0           & 0           \\
    \multicolumn{1}{l|}{\textbf{d}} & 0           & 1           & 1           & -1          & 0           & 1           & 0           \\
    \multicolumn{1}{l|}{\textbf{e}} & 0           & 0           & 1           & 1           & -1          & 1           & 0
    \end{tabular}
    \caption{The values for $x_{vf}$. For example, $v_1$ has inner turn in the boundary cycle $b$.}
    \label{table:values_xvf}
  \end{table}

  \begin{table}[h]
    \centering
    \begin{tabular}{llllll}
                                    & \textbf{a} & \textbf{b} & \textbf{c} & \textbf{d} & \textbf{e} \\ \cline{2-6}
    \multicolumn{1}{l|}{\textbf{a}} & 0          & 2          & 1          & 0          & 4          \\
    \multicolumn{1}{l|}{\textbf{b}} & 0          & 0          & 1          & 1          & 0          \\
    \multicolumn{1}{l|}{\textbf{c}} & 0          & 1          & 0          & 0          & 0          \\
    \multicolumn{1}{l|}{\textbf{d}} & 0          & 1          & 0          & 0          & 0          \\
    \multicolumn{1}{l|}{\textbf{e}} & 0          & 0          & 0          & 2          & 0
    \end{tabular}
    \caption{The values for $z_{fg}$. For example, the number of inner turns in the breakpoints of the edges that $d$ shares with $b$ is 2.}
    \label{table:values_zfg}
  \end{table}

\subsection*{Exercise 2.2}
  
  The sum of all the turns that $f$ makes in its vertices is:
  $$\sum_{f \in F, v \in V}{x_{vf}}$$
  The sum of all the inner turns that $f$ makes in its breakpoints is:
  $$\sum_{f \in F}\sum_{g \in F \setminus f}z_{fg}$$
  If there exists an inner turn from $g$ to $f$ it must mean that that turn is an outer turn seen from $f$ to $g$ so we can therefore sum the outer turns up as following:
  $$\sum_{f \in F}\sum_{g \in F \setminus f}z_{gf}$$
  which can be combined to a single sum:
  $$\sum_{f \in F}\sum_{g \in F \setminus f}(z_{fg} - z_{gf})$$
  We then combine these equations and state them as the following constraints:
  \begin{align*}
      \sum_{f \in F, v \in V}(x_{vf}) + \sum_{f \in F}\sum_{g \in F \setminus f}(z_{fg} - z_{gf}) = \begin{cases}
                                                                                       4, & \text{if } f \text{ is internal}\\
                                                                                      -4, & \text{otherwise}
                                                                                   \end{cases}
  \end{align*}

  Finally, let's verify that the constraints hold for faces $a$ and $e$ from Figure 3 in the assignment description. We lookup the values in Table \ref{table:values_xvf} and \ref{table:values_zfg}:
  \begin{align*}
    a &: x_{v_{3}a} + x_{v_{5}a} + x_{v_{6}a} + ((z_{ab} - z_{ba}) + (z_{ac} - z_{ca}) + (z_{ae} - z_{ea}))\\
      &= 1 + 1 + 1 + ((0 - 2) + (0 - 1) + (0 - 4)) \\
      &= -4
  \end{align*}
  \begin{align*}
    e &: x_{v_{3}e} + x_{v_{4}e} - x_{v_{5}e} + x_{v_{6}e} + ((z_{ea} - z_{ae}) + (z_{ed} - z_{de}))\\
      &= 1 + 1 -1 + 1 + ((4 - 0) + (0 - 2)) \\
      &= 4
  \end{align*}

\subsection*{Exercise 2.3}
It is a necessary but not sufficient condition for $G$ to be a rectilinear layout that no vertex has degree greater than 4. We see that this is the case by looking at the different ways in which we can connect edges to a vertex. Given the constraints that edges mustn't cross, and that they must be either vertical or horizontal, we find that there are only four ways to connect an edge to a vertex, namely the four cardinal directions.\\
\\
The restrictions on edges already mentioned also mean there are at most four different ways in which we can connect between two and four edges to a vertex, considering rotation of the vertex and edges not to have any meaning. These four ways to connect edges are shown on Figure \ref{fig:e1_a_solution}.

\begin{figure}[h]
    \centering
      \includegraphics[width=0.4\textwidth]{figures/e2_3}
    \caption{The four different ways to connect edges to a vertex, given the vertical/horizontal and non-crossing edges constraints.}
    \label{fig:e2_3}
\end{figure}

\subsection*{Exercise 2.4}

  \begin{equation*}
    \begin{aligned}
    & {\text{minimize}}
    & & \sum_{f \in F}\sum_{g \in F \setminus f} z_{fg} \\
    & \text{subject to}
    & & \sum_{f \in F, v \in V}(x_{vf}) + \sum_{f \in F}\sum_{g \in F \setminus f}(z_{fg} - z_{gf}) = \begin{cases}
                                                                                      4, & \text{if } f \text{ is internal}\\
                                                                                      -4, & \text{otherwise}
                                                                                     \end{cases} \\
    & & &  \sum_{f \in F}{x_{vf}} = \begin{cases}
                                        0 & \text{if v has degree 2}\\
                                        2 & \text{if v has degree 3}\\
                                        4 & \text{if v has degree 4}
                                    \end{cases}\\
    & & & z_{fg} \geq 0
    \end{aligned}
  \end{equation*}

\subsection*{Exercise 2.5}

We have:

\begin{align*}
      \sum_{f \in F, v \in V}(x_{vf}) + \sum_{f \in F}\sum_{g \in F \setminus f}(z_{fg} - z_{gf}) = \begin{cases}
                                                                                       4, & \text{if } f \text{ is internal}\\
                                                                                      -4, & \text{otherwise}
                                                                                   \end{cases}
\end{align*}
Making the following substitution for $x_{vf}$ :
\begin{align*}
  x_{vf} =  \begin{cases}
               1, & \text{if } f \text{ makes an inner turn in } v \\
               0, & \text{otherwise.}
            \end{cases}
\end{align*}
we can write:

\begin{align*}
  \sum_{f \in F, v \in V}{x_{vf}^{-}} + \sum_{f \in F, v \in V}{x_{vf}^{+}} + \sum_{f \in F}\sum_{g \in F \setminus f}(z_{fg} - z_{gf}) = \begin{cases}
                                                                                       4, & \text{if } f \text{ is internal}\\
                                                                                      -4, & \text{otherwise}
                                                                                   \end{cases}
\end{align*}
where $x_{vf}^{-}$ represents an inner turn and $x_{vf}^{+}$ represents an outer turn. 

We then rewrite: (TODO fix alignment problem)
\begin{align*}
  &= \sum_{f \in F, v \in V}{x_{vf}^{-}} + \sum_{f \in F, v \in V}{x_{vf}^{+}} + \sum_{f \in F}\sum_{g \in F \setminus f}(z_{fg} - z_{gf})\\
  &= \sum_{f \in F, v \in V}{x_{vf}^{-}} + \sum_{f \in F, v \in V}{x_{vf}^{+}} + \sum_{f \in F, g \in F\setminus f}{z_{fg}} - \sum_{f \in F, g \in F \setminus f}z_{gf}\\
  &= \sum_{f \in F, v \in V}{x_{vf}^{-}} + \sum_{f \in F, g \in F\setminus f}{z_{fg}} - \sum_{f \in F, v \in V}{x_{vf}^{+}} + \sum_{f \in F, g \in F \setminus f}z_{gf}\\
\end{align*}

Since $x_{vf}^{-}$ represents an inner turn of $f$ in $v$ and $z_{fg}$ represents the number of inner turns that $f$ makes in the breakpoints between $f$ and $g$, the summation of $x_{vf}^{-}$ and $z_{fg} $is the total number of inner turns in $f$ which we will call something else.... ?

The vertices in the MCFP graph will consist of the faces \textit{and} the original vertices of the planar embedding. The edges in the flow graph will be between:

\begin{enumerate}[(i)]
  \item faces which are adjacent in the planar embedding
  \item vertices in the planar embedding which lie on a boundary cycle of a face
\end{enumerate}

The demand for a vertex $v$ if the vertex represents a face in the planar embedding is given by the formula:

\begin{align*}
    demand(v) = \begin{cases} 
                  4  & \text{if } v \text{ represents a internal face}\\ 
                  -4 & \text{if } v \text{ represents a external face}
                \end{cases}
\end{align*}

The demand for a vertex $v$ if $v$ represents an original vertex in the planar embedding is given by the formula:
\begin{align*}
    demand(v) = \begin{cases} 
                  0  & \text{if } v \text{ has degree 2} \\ 
                  -2 & \text{if } v \text{ has degree 3} \\ 
                  -4 & \text{if } v \text{ has degree 4} 
                \end{cases}
\end{align*}

The capacities between faces which share an edge in the flow graph are unbounded with a cost of 1. The capacities are unbounded because the number of breakpoints between faces is unlimited. The capacity of edges starting in an original vertex is 1 because it can provide at most 1 inner turn to a face. The cost on these edges is none since there is a vertex lying on these kind of breakpoints.

Figure \ref{fig:mcfp_instance} shows an instance of a MCFP for the original planar embedding shown in figure \ref{fig:planar_embedding}. The red edges have cost 1 and the black edges have cost 0. The flow across a red edge represents the number of breakpoints between adjacent faces in the rectilinear layout. A flow across a black edge represents an vertex $v$ making an inner turn in a face $f$.

\begin{figure}[h]
  \centering
    \includegraphics[width=0.6\textwidth]{figures/e2_5_planar_embedding}
  \caption{The original planar embedding.}
  \label{fig:planar_embedding}
\end{figure}

\begin{figure}[h]
  \centering
    \includegraphics[width=0.8\textwidth]{figures/e2_5}
  \caption{An instance of MCFP corresponding to the embedded graph in Figure \ref{fig:planar_embedding}}.
  \label{fig:mcfp_instance}
\end{figure}

\section{3}
\subsection{3.1}
\subsection{3.2}
\subsection{3.3}
\subsection{3.4}

\end{document}
% End of document.

