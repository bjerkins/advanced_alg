\documentclass[12pt]{article}
\usepackage[T1]{fontenc}
\usepackage[utf8]{inputenc}
\usepackage{url}
\usepackage{enumerate}
\usepackage[top=3cm, bottom=3cm]{geometry}
\usepackage{graphicx} 
\usepackage{enumitem}
\usepackage{natbib}
\usepackage{listings}
\usepackage{float}
\usepackage{amsmath}
\usepackage{color}
\bibpunct{[}{]}{,}{a}{}{;}
\setcitestyle{super}

% Variables
\newcommand{\assignmentname}{Assignment 1}
\newcommand{\coursename}{Advanced Algorithms and Data Structures}
\newcommand{\studentname}{Bjarki Madsen (lch929) - Michael Bang (tqg432)}
\newcommand{\department}{Department of Computer Science}
\newcommand{\institution}{Copenhagen University}
\newcommand{\location}{Copenhagen, Denmark}

\begin{document}

\renewcommand\refname{References}

\title{\assignmentname \\ {\Large {\textsc \coursename}}}
\author{
        \studentname \\
        \department \\
        \institution \\
        \location
}
\date{\today}

\maketitle
\thispagestyle{empty}

\pagebreak

\section*{Exercise 1}

  Figure \ref{fig:e1_a_solution} shows a b-flow with maximum flow value of 9 for the first proposed graph from the assignment. The second proposed graph from the assignment has no b-flow. The reason is that $v_1$ wants to have at least $3 + c(v_2, v_1)$ after giving away $f(v_1, v_3)$. In order for that to happen the $f(v_2, v_1)$ has to equal to $c(v_2, v_1)$. After the $f(v_2, v_1)$, $v_2$ wants to end up with 1 in demand, meaning it needs a flow value of 4 to itself but that is impossible because the maximum flow to $v_2$ is equal to $c(v_3, v_2)$, or 2 and therefore is unable to fulfill $f(v_2, v_1) = c(v_2, v_1)$. 

  \begin{figure}[h]
    \centering
      \includegraphics[width=0.6\textwidth]{figures/e1_a_solution}
    \caption{Graph showing b-flow with maximum flow = 9}
    \label{fig:e1_a_solution}
  \end{figure}

\section*{Exercise 2}
\subsection*{Exercise 2.1}

  Tables \ref{table:values_xvf} and \ref{table:values_zfg} show the values for $x_{vf}$ and $z_{fg}$, respectively. The number of breakpoints are 13.
  
  \begin{table}[h]
    \centering
    \begin{tabular}{llllllll}
                                    & \textbf{v1} & \textbf{v2} & \textbf{v3} & \textbf{v4} & \textbf{v5} & \textbf{v6} & \textbf{v7} \\ \cline{2-8} 
    \multicolumn{1}{c|}{\textbf{a}} & 0           & 0           & 1           & 0           & 1           & 1           & 0           \\
    \multicolumn{1}{l|}{\textbf{b}} & 1           & 0           & 0           & 0           & 0           & 1           & 0           \\
    \multicolumn{1}{l|}{\textbf{c}} & 1           & 1           & 1           & 0           & 0           & 0           & 0           \\
    \multicolumn{1}{l|}{\textbf{d}} & 0           & 1           & 1           & -1          & 0           & 1           & 0           \\
    \multicolumn{1}{l|}{\textbf{e}} & 0           & 0           & 1           & 1           & -1          & 1           & 0          
    \end{tabular}
    \caption{The values for $x_{vf}$. For example, $v_1$ has inner turn in the boundary cycle $b$.}
    \label{table:values_xvf}
  \end{table}

  \begin{table}[h]
    \centering
    \begin{tabular}{llllll}
                                    & \textbf{a} & \textbf{b} & \textbf{c} & \textbf{d} & \textbf{e} \\ \cline{2-6} 
    \multicolumn{1}{l|}{\textbf{a}} & 0          & 2          & 1          & 0          & 4          \\
    \multicolumn{1}{l|}{\textbf{b}} & 0          & 0          & 1          & 1          & 0          \\
    \multicolumn{1}{l|}{\textbf{c}} & 0          & 1          & 0          & 0          & 0          \\
    \multicolumn{1}{l|}{\textbf{d}} & 0          & 1          & 0          & 0          & 0          \\
    \multicolumn{1}{l|}{\textbf{e}} & 0          & 0          & 0          & 2          & 0         
    \end{tabular}
    \caption{The values for $z_{fg}$. For example, the number of inner turns between cycle $b$ and $c$ is 1.}
    \label{table:values_zfg}
  \end{table}

\subsection*{Exercise 2.2}
  
  \begin{align*}
      \sum_{f \in F, v \in V}(x_{vf}) + \sum_{f \in F}\sum_{g \in F \setminus f}(z_{fg} - z_{gf}) = \begin{cases}
                                                                                       4, & \text{if internal}\\
                                                                                      -4, & \text{if external}
                                                                                   \end{cases}
  \end{align*}

\subsection*{Exercise 2.3}
It is a necessary but not sufficient condition for $G$ to be a rectilinear layout that no vertex has degree greater than 4. We see that this is the case by looking at the different ways in which we can connect edges to a vertex. Given the constraints that edges mustn't cross, and that they must be either vertical or horizontal, we find that there are only four ways to connect an edge to a vertex, namely the four cardinal directions.\\
\\
The restrictions on edges already mentioned also mean there are at most four different ways in which we can connect between two and four edges to a vertex, considering rotation of the vertex and edges not to have any meaning. These four ways to connect edges are shown on Figure \ref{fig:e1_a_solution}. 

\begin{figure}[h]
    \centering
      \includegraphics[width=0.4\textwidth]{figures/e2_3}
    \caption{The four different ways to connect edges to a vertex, given the vertical/horizontal and non-crossing edges constraints.}
    \label{fig:e2_3}
\end{figure}

\subsection*{Exercise 2.4}
  
  \begin{equation*}
    \begin{aligned}
    & {\text{minimize}}
    & & \sum_{f \in F}\sum_{g \in F \setminus f} z_{fg} \\
    & \text{subject to}
    & & \sum_{f \in F, v \in V}(x_{vf}) + \sum_{f \in F}\sum_{g \in F \setminus f}(z_{fg} - z_{gf}) = \begin{cases}
                                                                                      4, & \text{if internal}\\
                                                                                      -4, & \text{if external}
                                                                                     \end{cases} \\
    & & &  \sum_{f \in F}{x_{vf}} = \begin{cases}
                                        0 & \text{if v has degree 2}\\
                                        2 & \text{if v has degree 3}\\
                                        4 & \text{if v has degree 4}
                                    \end{cases}\\
    & & & z_{fg} \geq 0
    \end{aligned}
  \end{equation*}

\subsection*{Exercise 2.5}

\end{document}
% End of document.







